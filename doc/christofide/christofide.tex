\documentclass{article}

\usepackage{algorithm}
\usepackage{algpseudocode}

\usepackage{enumitem}
\usepackage{amsmath, amssymb}

\begin{document}

\section{Christofide}
\subsection{Background}
The Christofides algorithm improves on the 2-approximation algorithm, providing
a worst case ratio of $3/2$ at the cost of increasing the running time to
$O(n^3)$ [1].\\

It can be summarized with the following steps:
\begin{itemize}
  \item Using the entire graph, build a minimum-spanning tree Make a set of
        vertices
  \item Make a set of vertices with odd degree
  \item From the set of vertices with odd degree, make a minimum weight matching
  \item Add the minimum weight matching to the minimum-spanning tree
  \item Find a Euler tour from the graph made by combining the minimum spanning
        tree with minimum weight matching
  \item Remove repeated vertices, thus turning it into a Hamiltonian circuit [1]
\end{itemize}

Step one, getting the minimum spanning tree, makes it such that we are working
with a new graph that has eliminated some of the higher cost edges without
getting rid of any of the vertices we must visit (recall all must be visited).
Next, we get the set of vertices with an odd degree which, by the handshaking
lemma, must be an even number of vertices [2]. Finding a minimum weight matching
on these and adding it to the minimum spanning tree ensures that the shortest
of edges will be available when finding the tour. Finding a minimum weight
matching also ensures each vertex has enough edges to avoid having to reuse
edges in order to get off a vertex. In other words, it ensures a Euler tour
can be found on this subgraph that includes all the lowest-weight edges. After
finding a Euler tour, we can simply remove repeated vertices because each vertex
can only be visited once, and in the actual map all vertices can be reached. Due
to the triangle inequality, removing repeated vertices doesn’t increase weight.
[2] Tests have shown that the algorithm “tends to place itself 10\% above the
Held-Karp lower bound”[1].\\

The key difference between it and the 2-approximation
algorithm is that, rather than simply duplicating edges to ensure an Euler
cycle, it creates a minimum weight matching on the vertices with odd degree and
combines it with the minimum spanning tree [1].

\subsection{pseudocode}
Below in the pseudocode for the algorithm. Note that it combines the step of
finding a minimum weight matching with the one that combines it with the minimum
spanning tree.

\begin{algorithm}
  \caption{Christofide}
  \label{alg1}
  \begin{algorithmic}[1]
    \Procedure {Christofide}{$\{G=(V,E),\ s\}$}

    \For {$v \in V$}
      \State $v_{\textit{key}} \gets \infty$
      \State $v_{\textit{parent}} \gets \varnothing$
    \EndFor

    \State $s_{\textit{key} = 0}$

    \While {$|k|$}

    \EndWhile

    \EndProcedure
  \end{algorithmic}
\end{algorithm}

\subsection{Works Cited}
\end{document}
